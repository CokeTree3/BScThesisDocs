\section{Discussion}\label{s:discussion}

%%A “limitations” section, as the name implies, describes scenarios where the
%%proposed solution may not work well. Although a “discussion” section could also
%highlight limitations of the proposed work, it focuses on analyzing the
%implications of the proposed work for current and future research.

\subsection{Design Limitations}
The implemented detection mitigations are system wide, this provides the best level of transparency, even in the case of an attempt 
to perform these detection checks on a different process(cite this). Downside of this approach is that it will also interfere with normal, 
genuine processes that expect certain data or values to be present, like the VM support programs many commercial hypervisor vendors use, 
which might store crucial data in locations blocked by these transparency features, as well as Windows processes that expect no kernel modifications. 
As an addition to the transparency mode in HyperDbg, a 2 stage transparency, with a semi-transparent mode added, would improve the usability and reliability of the hypervisor. 
In this system, a select number of processes are allowed to bypass all transparent mitigations as well as not intercepting the MSR accesses that could cause instability. 
As a tradeoff, this semi-transparent mode would allow malware to more easily run successful hypervisor detections and avoid full analysis, by emulating the appearance of one 
of these genuine, whitelisted processes.

\subsection{Not Full Transparency}
This paper only covered 3 general hypervisor detection vectors, there are numerous other ways any user-mode process could detect the virtualized environment, 
and a fully transparent reliable hypervisor is nearly impossible to create. Deploying HyperDbg with the transparency features from this paper does not guarantee any true 
transparency and it can remain stealthy if the malware or any other binary analyzed only uses the detection methods discussed in this paper.

The implemented transparency features also leave detectable traces of their own and a process developed with HyperDbg in mind could find these 
traces and still detect the virtualized environment of HyperDbg’s hypervisor. The remaining artifacts include inconsistent return values, memory buffers 
that are not the correct size or inconsistencies between the same data obtained from different sources, like a system call which does not get intercepted by HyperDbg.



%%% Local Variables:
%%% mode: latex
%%% TeX-master: "../thesis"
%%% End:
