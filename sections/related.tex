\section{Related Work}\label{s:related}

Research into transparent hypervisors that could evade detection from the user-space and allow improved dynamic malware analysis have been a focus point 
of security research for quite some time. \stress{Ether}, as proposed by Dinaburg et al.~\cite{ether} was one of the first malware analysis designs that took 
advantage of virtualization, hiding the analysis tools from the system. But \stress{Ether} did not employ techniques to hide its hypervisor from the system, as proposed in this paper, 
so as Pek et al.~\cite{n-eteher} proves, \stress{Ether} could still be easily detected if the process attempting the detection knew what to look for. Shi et al.~\cite{apate} improves 
the design by implementing some mitigations to hypervisor detections in \stress{Apate}, which achieved much better transparency, but mostly focused on anti debugging methods.

The use of hypervisors to evade detections in user-space is not an approach only used by malware analysts, the malware itself, in the form of a hypervisor, can deploy itself 
and evade traditional anti-virus detections. King et al.~\cite{1624022} in 2006 proposed the virtual-machine based rootkit (VMBR) malware type, which virtualizes the infected system 
and isolates themselves from the running system. An example of a VMBR being \stress{BluePill}, as proposed by Rutkowska~\cite{rutkowska2006introducing} in 2006. 
And while \stress{BluePill} did not attempt hypervisor detection evasion, a big focus in security research has not been on just hiding hypervisors from malware, 
but also on detecting these stealthy hypervisors themselves. In 2015, Korkin~\cite{stealthy-hv-detection} proposed 2 reliable ways to detect stealthy hypervisors, 
based around instruction execution time analysis. This approach is similar to the RDTSC instruction based time detection as proposed in section~\ref{HV_detection}.
Similarly, Brengel et al.~\cite{brengel2016detecting} look into more timing related hypervisor detection methods and show that malware can use these detection methods to evade analysis.

%%% Local Variables:
%%% mode: latex
%%% TeX-master: "../thesis"
%%% End:
