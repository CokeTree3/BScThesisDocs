\section{Introduction}\label{s:intro}

Modern malware is becoming increasingly more complex and advanced.
Its internals are constantly changing and so the ability to understand the inner workings and develop protections against it is an increasingly difficult task. 
Compiled malware binaries are frequently obfuscated, making static analysis hard, so dynamic analysis is preferred~\cite{Leon2021, 10.1145/2245276.2232070}. However, performing dynamic analysis requires a safe environment 
to deploy the malware. One commonly used method for dynamic analysis and malware detection is to deploy the malware inside a virtual machine, a virtualized environment, managed by a virtual machine monitor (VMM),
commonly known as a hypervisor. This adds an abstraction layer over hardware, isolating the environment from direct hardware access and from any other environments running on the same bare metal system. 
Modern virtualization relies on hardware level CPU instruction set extentions implemented by both Intel\textsuperscript{\tiny\textregistered} VT-x\texttrademark and AMD\textsuperscript{\tiny\textregistered} AMD-V\texttrademark. 
These features allow automation of low-level virtualization elements and a more direct access to the hardware from the virtualized environment, allowing levels of performance that make these environments usable.

Virtualization and the use of hypervisors is a useful tool for process disassembly and analysis, as it allows kernel-level view of the execution flow and direct access to certain instructions that are executed on the CPU, as well as almost full
control of the virtual CPU.
Another benefit of this hypervisor-assisted malware analysis is that hypervisors can operate externally, at a lower privilege level than the guest operating system, refered by Intel as VMX root mode~\cite{Intel-SDM2025}.
System analysis at this privilege level, sometimes refered to as ring level -1,  is by itself considerably more transparent and harder to detect from the user space.

But it is not impossible. There exist many different methods any process operating at the user-space level can detect that the environment is virtualized. 
Some detection methods involve specific CPU instruction and feature tests, some use precise timing data between events, and others might use the register and memory access patterns. 
Malware and other programs can use these techniques to detect the presence of a hypervisor and choose to execute different code or even nothing at all~\cite{10.1109/TIFS.2020.2976559}. 
In the case of malware, this can mean that when an attempt is made to inspect the execution of the malicious code within the virtualized environment, 
no malicious code is actually deployed and the malware can stay hidden, evading any attempts at disassembly or detection.

A vector more advanced hypervisor detection methods use involves testing for specific hypervisor vendors. Most of the commonly used consumer hypervisors, like VirtualBox\texttrademark and VMWare Workstation\texttrademark, 
add drivers, files, and other data inside the guest operating systems, mostly to improve performance or compatibility, especially on frequently used and popular operating systems like Microsoft Windows. 
All of this data is in some way necessary or convenient, but for malware analysis, the presence of such data is just another way that malware can reliably detect the virtual environment and evade the analysis attempts.

From the hypervisor's point of view, attempts to detect its presence can also be detected. The low-level position of hypervisors allows them to not just observe all execution happening on the guest system, 
but also inject code into the guest and change its behavior. This allows mitigations to exist that can defeat most attempts at detecting the virtual environment.

In this thesis we research hypervisor detection methods and design mitigations for these methods within a hypervisor. To implement these hypervisor detection mitigation features, 
the hypervisor based debugging tool HyperDbg was chosen as the platform. HyperDbg is a hypervisor-based Windows debugging tool that provides kernel-level debugging of a virtualized system which attempts to be as transparent to the guests as possible.
While there are many different ways processes can detect the virtualized environment they are deployed in, the design developed in this thesis only focuses on two general detection vectors, 
CPU-level instructions and structures, and guest OS system calls.

The mitigation techniques designed were implemented into HyperDbg's transparency mode. 
These techniques were then tested against a selection of publicly available virtual machine detection tools that combine many of the common methods for detecting hypervisors into one program. 
The evaluation also includes analysis of the overhead introduced by the implementation, as well as analysis of the efficiency of the system in evading detection attempts when compared to ScyllaHide~\cite{scyllahide}, an anti-anti-debugging tool, which features a similar implementation in evading debugger detection.
The research of this thesis focuses on the Microsoft Windows operating system, as at the time of writing this project, Windows was the only guest OS supported by HyperDbg.

The contributions made in this thesis are as follows:

\case{}
Design of hypervisor and debugger detection mitigation features that obscure attempts at detecting the presence of a virtualized environment.

\case{}
Implementing a design of Windows system call interception that cannot be bypassed with direct system calls. 

\case{}
An evaluation on feasibility of these features at successfully hiding the hypervisor and evading anti-hypervisor detections.

\case{}
An implementation of these features into an open-source hypervisor in a modular and dynamic transparency mode.


%%% Local Variables:
%%% mode: latex
%%% TeX-master: "../thesis"
%%% End:
