%%%%%%%%%%%%%%%%%%%%%%%%%%%%%%%%%%%%%%%%%%%%%%%%%%%%
% Artifact Appendix Template for Usenix Security'24 AE
%%%%%%%%%%%%%%%%%%%%%%%%%%%%%%%%%%%%%%%%%%%%%%%%%%%%

\section{Artifact Appendix}

%%%%%%%%%%%%%%%%%%%%%%%%%%%%%%%%%%%%%%%%%%%%%%%%%%%%%%%%%%%%%%%%%%%%%
\subsection{Abstract}
{\em During this thesis, multiple hypervisor detection mitigation approaches were discussed and designed. These mitigation methods were then implemented into HyperDbg, 
developing the transparency mode of HyperDbg. The transparency features attempt to hide the HyperDbg hypervisor and any other virtualized environment HyperDbg has been nested on top of, 
by intercepting and modifying the data a calling user process will receive and are deployed from HyperDbg during runtime without requiring system reboot.}

%%%%%%%%%%%%%%%%%%%%%%%%%%%%%%%%%%%%%%%%%%%%%%%%%%%%%%%%%%%%%%%%%%%%%
\subsection{Description \& Requirements}

\textit{Both the development and evaluation of the hypervisor detection technique mitigation methods were carried out
on a Microsoft Windows 11 24H2 bare metal system as the host system, with a 10th generation Intel Core\textsuperscript{TM} CPU which supports VT-x hardware virtualization feature.
The virtualized guest system was run on this host using a VMWare Workstation 16 hypervisor with nested virtualization support enabled.
This guest system was used to run HyperDbg's hypervisor and perform all evaluation and testing. The development of the transparency features in HyperDbg was based on its 0.13.0 release version. }

\subsubsection{Security, privacy, and ethical concerns}
\textit{The containerized nature of virtual machines adds a layer of security to a bare metal system that is used as the host for malware analysis.
During this project, no actual malware samples were tested, only non-malicious parts of them, so full host system safety cannot be guaranteed from this thesis.
To be able to run HyperDbg's hypervisor, Windows Driver Signature Enforcement security feature as well as VBS(Virtualization Based Security) must be disabled on the guest system.
This allows the loading of unsigned drivers, which HyperDbg's hypervisor driver is, but it also allows any other process to load their unsafe drivers, 
exposing the system to a much lower level and harder to detect malware attacks. HyperDbg itself, including the features implemented as part of this thesis, are open source and available under the GNU GPL v3.0 copyleft license.}

\subsubsection{How to access}
\textit{Access to the developed transparency features is freely available on GitHub, as part of the HyperDbg project~\cite{hyperdbg}. 
The individual commits to intermediate feature implementations are available at: CPUID transparency \href{https://github.com/HyperDbg/HyperDbg/commit/56c2d400fddca22f314c1e5d16cadce6950eb262}{Link}, 
MSR transparency \href{https://github.com/HyperDbg/HyperDbg/commit/2cc3da3daebd793fd56d5c370ef6da398efa6d29}{Link}, Windows system call
transparency \href{https://github.com/HyperDbg/HyperDbg/commit/d0610661baf944259c33e921d3fc265e6a217ad8}{Link}
}


\subsubsection{Hardware dependencies}
\textit{To be able to evaluate the hypervisor transparency features implemented as part of this thesis.
The system running the hypervisor driver must have an Intel x64 CPU which supports VT-x and EPT technologies. As of writing this thesis, AMD CPUs are not supported.
Both the host and guest systems should also have at least a serial port each that are connected together, this can be a physical connection or a virtual one in case of nested virtualization usage.}

\subsubsection{Software dependencies}
\textit{Both the host and guest systems must be running the 64 bit version of Microsoft Windows operating system,
a certain release version is not specified, but testing was done, and correct operations can only be guaranteed, on Windows 11, with most of the functionality also remaining in Windows 10.
HyperDbg can be obtained as a compiled software, so no compilation is needed. To be able to load the hypervisor driver of HyperDbg,
Driver Signature Enforcement feature needs to be disabled, one of the methods, and the one that was used during the evaluation section, involves the use of a third party tool WinDbg \href{https://github.com/HyperDbg/HyperDbg/commit/2cc3da3daebd793fd56d5c370ef6da398efa6d29}{Link}, but this is not required.}

\subsubsection{Benchmarks}
\textit{None}

%%%%%%%%%%%%%%%%%%%%%%%%%%%%%%%%%%%%%%%%%%%%%%%%%%%%%%%%%%%%%%%%%%%%%
\subsection{Set-up}

\textit{For this example evaluation, the nested virtualization approach is chosen, in this case the guest system on which the evaluation will be performed,
is a VMWare Workstation 16 virtual machine running Microsoft Windows 11 24H2. This is assumed to already be set up and in a working state. The precompiled HyperDbg package was used.}

\subsubsection{Installation}
\textit{To deploy the transparency features, only the guest VMWare system and HyperDbg is needed. The downloaded compiled HyperDbg package directory needs to be present on both the guest and host systems. 
Once HyperDbg has been deployed in kernel debug mode, following its installation guide available at: \href{https://docs.hyperdbg.org/getting-started/build-and-install}{Link}, the transparency mode, 
which contains the features developed, can be enabled by entering the "!hide" command in the CLI of the host system, and once the confirmation output is displayed, the hypervisor transparency has been enabled 
and the control can be returned to the guest with the "g" command. Now any user-space process attempting to call the CPUID instruction or querying one of the intercepted system calls 
will receive the mitigated output.}

\subsubsection{Basic Test}
\textit{The functionality of the hypervisor detection mitigation features can be tested by creating and running a simple program that queries the CPUID instruction with the EAX register set to 0x01
 and then reading the 31st bit of the returned value in RAX. This is the "hypervisor present" bit and in a normal virtual environment this bit would be set, but the transparency features unsets it, 
 to appear as if this is a bare metal system.}

%%%%%%%%%%%%%%%%%%%%%%%%%%%%%%%%%%%%%%%%%%%%%%%%%%%%%%%%%%%%%%%%%%%%%
\subsection{Evaluation workflow}

\subsubsection{Major Claims}
The major claims made during this thesis.\\

\begin{compactdesc}

    \item[(C1):] \textit{A virtual system with the hypervisor detection mitigations enabled is more transparent and the hypervisor is harder to detect from the user-space. 
    This is proven in section~\ref{HV-detection-eval} by evaluating the designed mitigation system against third party tools that execute commonly used hypervisor detection techniques. 
    The improvement can be seen in Figure~\ref{fig:test_accuracy}}

    \item[(C2):] \textit{The designed system for Windows system call interception is more resilient than other approaches used by similar implementations like ScyllaHide. Section~\ref{scyllahide-comparison}
    proves that intercepting system calls at the kernel level improves the resiliency of the mitigations. 
    A process making a system call directly to the kernel bypasses the interception attempts in the design of ScyllaHide, but the system designed in this thesis, in Section~\ref{syscall_interception} can still intercept it}

\end{compactdesc}

\subsubsection{Experiments}
\textit{The experiments performed for the evaluation of the designed transparency features. Corresponding to the major claims made in the previous section.
For all of the experiments below, HyperDbg is deployed in kernel debug mode and active during them, with the transparency mode disabled, unless stated otherwise.}

% use paralist for more compact list format: for more details check here:
% https://texfaq.org/FAQ-complist
\begin{compactdesc}

    \item[(E1):] \textit{Evasion efficiency:
    The experiment consisted of running one of the 3 chosen hypervisor detection tool suites on the virtualized system running HyperDbg with the transparency mode disabled as well as enabled}

    \begin{asparadesc}
        \item[How to:]  \textit{Once HyperDbg has been successfully deployed on the guest system, the detection suite is run on the guest. The result given by the tools vary, but all of them will contain 
        a number of positive detections over the total count of detection methods performed. This score can then be logged and the transparency mode can be enabled. Once it is enabled, the same tool can be run again 
        and the, now much smaller, score can also be logged and compared to the previous one.}

        \item[Results:] \textit{The results should show a decrease in positive detections when they were performed with the transparency mode enabled. For a more clear comparison, the two percentage values can also
        be plotted, as it can be seen in Figure~\ref{fig:test_accuracy}.}
    \end{asparadesc}

    \item[(E2):] \textit{Syscall interception:
    For this experiment a custom program needs to be created that calls a specific Windows system call in two different ways and displays the return value.}

    \begin{asparadesc}
        \item[How to:]  \textit{This experiment requries the creation of a small program that calls the NtSystemDebugControl systemcall in two different ways, once by calling the NT API function of the same name
        and then calling directly to the kernel by executing the SYSCALL assembly instruction like seen in Listing~\ref{lst:syscall-wrapper}.}
        
        \item[Preparation:]  \textit{To compare the design of the mitigations, ScyllaHide needs to be obtained from its source and made available to the guest system. The same goes for any debugging tool that can
        can be used to attach itself to the process of the testing program. The aforementioned program also needs to be created and available.} 
        
        \item[Execution:]  \textit{Firstly, ScyllaHide can be deployed, by attaching its library to the process of the testing program, using the provided CLI. Then the debugger should also be attached to the same 
        process. The system call test can then be executed, and after the returned values are shown, it should be noticeable  that ScyllaHide can only intercept the API based call and not the direct kernel system call.
        After this the same process can be repeated without attaching ScyllaHide, but enabling the transparency mode in HyperDbg.}
        
        \item[Results:] \textit{The results should show ScyllaHide only intercepting and mitigating the first approach of the system call calling, while HyperDbg's transparency mode is able to intercept and mitigate both
        calling approaches.}
    \end{asparadesc}

    \item[(E3):] \textit{[Optional Name] [1 human-hour + 3 compute-hour]: ...}

\end{compactdesc}

%%%%%%%%%%%%%%%%%%%%%%%%%%%%%%%%%%%%%%%%%%%%%%%%%%%%%%%%%%%%%%%%%%%%%
\subsection{Notes on Reusability}
\label{sec:reuse}

\textit{The hypervisor detection mitigation features, as implemented into HyperDbg, can allow more reliable dynamic malware analysis when using HyperDbg.}

\textit{These features, developed as part of this thesis, are only a small portion
of the numerous ways that the presence of hypervisors can be detected. As with most cybersecurity development, 
full security and protection cannot be guaranteed. The development was done in a highly modular and expandable fashion, 
to allow future additions and improvements to hypervisor transparency in HyperDbg, as more mitigation methods are developed and new ways to detect hypervisors are discovered.}

%%%%%%%%%%%%%%%%%%%%%%%%%%%%%%%%%%%%%%%%%%%%%%%%%%%%%%%%%%%%%%%%%%%%%

\subsection{Version}
%%%%%%%%%%%%%%%%%%%%
% Obligatory.
% Do not change/remove.
%%%%%%%%%%%%%%%%%%%%
Based on the LaTeX template for Artifact Evaluation V20231005. Submission,
reviewing and badging methodology followed for the evaluation of this artifact
can be found at \url{https://secartifacts.github.io/usenixsec2024/}.